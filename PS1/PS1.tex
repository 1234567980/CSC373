\documentclass[12pt]{article}
\usepackage{dsfont}
\usepackage{amsmath}
\usepackage[margin=2.5cm]{geometry}
\usepackage{amssymb}
\usepackage{mathtools}

\DeclarePairedDelimiter\ceil{\lceil}{\rceil}
\DeclarePairedDelimiter\floor{\lfloor}{\rfloor}

\usepackage{algorithm}
\usepackage[noend]{algpseudocode}
\usepackage{pgf}
\usepackage{tikz}
\usepackage{tikz-qtree}

\title{\textbf{CSC373 - Problem Set 1}}
\author{Authors: Luke Bacchus, Naslin Rahman, Zhuoqian Li}
\date{\today}

\begin{document}
\maketitle
\section*{Question 1}
\begin{enumerate}
    \item[a.] We have $n$ piles of $m$ papers. As merging two sorted piles can be done it time proportional to the size of the resulting pile, following the given algorithm the first merge will take $2m$ time, the second merge will take $3m$ time, etc. 
    The running time of this algorithm can be found:
    \begin{align*}
        &= 2m + 3m + 4m + ... + nm \\
        &= \sum_{k=2}^n km \\
        &= m(\sum_{k=2}^n k) \\
        &= m(\sum_{k=1}^n k-1) \\
        &= m(\frac{(n+1)(n)}{2} - 1)
    \end{align*}
    This is $\mathcal{O}(mn^2)$.
    \item[b.] If you have exactly two piles, merge them as in a. If you have more than two piles, divide into two sets of piles of equal size (we can do this because $n$ is an exact power of two). 
    Then, recursively merge each set of piles until you obtain two piles $p1$ and $p2$. Then merge $p1$ and $p2$, and the run time of this process is proportional to the size of $p1 + p2$. The recurrence relation can be described by:
    
    \begin{equation}
        T(mn) = \begin{cases}
          0, & \text{n = 0 or m $=$ 0}.\\
          2T(\frac{mn}{2}) + mn, & \text{n $>$ 0 and m $>$ 0}.\\
    \end{cases}
    \end{equation}
By the master theorem, where in this case: a = 2, b = 2, d = 1 meaning that a = b. Thus, the (worst-case) running time is $O(mn \log mn)$.\end{enumerate}
\newpage
\section*{Question 2}
\begin{enumerate}
    \item[a.] Divide the array into two subarrays: $s1 = A[0:\frac{n}{2}]$ and $s2 = A[\frac{n}{2} + 1:n]$. Recursively find $S_{ij}$ such that $S_{ij} = \sum_{r=i}^j A[r]$ is the maximum sum over all possible $i$ and $j$, $1 \leq i \leq j \leq n$ for both subarrays. Then linearly scan left and right from the midpoint to find the maximum subarray beginning in $s1$ and ending in $s2$. Finally, take the maximum of all three maximum sums. 
    \begin{verbatim}
        # Assume arrays start at index 1
        function func(A, i, j):
            if i == j
                return A[i]
            mid = (i + j)/2
            leftSum = func(A, i, mid)
            rightSum = func(A, mid+1, j)
            crossSum = helper(A, i, j, mid)
        
            return max(leftSum, rightSum, crossSum)
    \end{verbatim}
    
    \begin{verbatim}
        function helper(A,start,end, mid):
            leftSum = 0
            rightSum = 0
            totalSum  = 0
        
            for i = mid downto start
                sum += A[i]
                if sum > leftSum
                    leftSum = sum
            
            sum = 0
            for j = mid +1 to end
                sum += A[j]
                if sum > rightSum
                    rightSum = sum
            
            return leftSum + rightSum
    \end{verbatim}
    
    The recurrence relation is $T(n) = 2T(n/2) + n$ for $n>0$. By the master theorm this is $O(nlogn)$
    
    \item[b.] Divide the array into two subarrays and recursively solve as in a. However, instead of just returning the maximum subarray, when we call our function, it will also return the maximum prefix and maximum suffix of the array. That is, the maximum prefix is $k$ such that the $\sum_{j=1}^k A_j$ is maximum; similarly the maximum suffix is $k$ such that the $\sum_{j=k}^n A_j$ is maximum. For our combine step, instead of linearly scanning over the array to find the maximum subarray beginning in $s1$ and ending in $s2$, we instead return the union of the maximum 
    suffix of $s1$ and maximum prefix of $s2$. Then, take the maximum of all three maximum sums.

    \begin{verbatim}
        # This function returns six values: 
        # totalSum sum of A from i to j, 
        # prefixSum value of max of prefix, 
        # suffixSum value of max suffix, 
        # subSum value of optimal subsequence in A[i] to A[j], 
        # prefixEnd index of last element of the maximum prefix, 
        # suffixStart index of first element of the maximum suffix
        
        function possibleSums(A, i, j):
            if i == j
                return (A[i], A[i], A[i], A[i], i, i)
            mid = floor((i + j)/2)
            left = possibleSums(A, i, mid)
            right = possibleSums(A, mid+1, j)

            # Sum of all elements in the array
            totalSum = left.totalSum + right.totalSum

            # Max possible prefix sum
            prefixSum = max(left.prefixSum, left.totalSum + right.prefixSum)

            # Max possible suffix sum
            suffixSum = max(right.suffixSum, left.suffixSum + right.totalSum)

            # Max subarray sum
            subSum = max(left.subSum, right.subSum, 
                         left.suffixSum + right.prefixSum) 

            ## Update left suffix and right prefix location
            # Default
            prefixEnd = left.prefixEnd
            suffixStart = right.suffixStart

            if (prefixSum == left.totalSum + right.prefixSum){
                prefixEnd = right.prefixEnd
            }
            if (suffixSum == left.suffixSum + right.totalSum){
                suffixStart = left.suffixStart
            }

            return (totalSum, prefixSum, suffixSum, subSum, prefixEnd, 
                suffixStart)
    \end{verbatim}

\end{enumerate}

\end{document}