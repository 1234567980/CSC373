\documentclass[12pt]{article}
\usepackage{dsfont}
\usepackage{amsmath}
\usepackage[margin=2.5cm]{geometry}
\usepackage{amssymb}
\usepackage{mathtools}

\DeclarePairedDelimiter\ceil{\lceil}{\rceil}
\DeclarePairedDelimiter\floor{\lfloor}{\rfloor}

\usepackage{algorithm}
\usepackage[noend]{algpseudocode}
\usepackage{pgf}
\usepackage{tikz}
\usepackage{tikz-qtree}

\title{\textbf{CSC373 - Problem Set 3}}
\author{Authors: Luke Bacchus, Naslin Rahman, Zhuoqian Li}
\date{\today}

\begin{document}
\maketitle
\section*{Question 1}
\begin{enumerate}
    \item[a.] Let A(i, h) = maximum grade one can receive on i projects if they spend h number of 
    hours on it. The Bellman equation is as follows. For each value of k between 0 and h we calculate 
    the max value of k that maximizes $f_i(k)$ and A[i-1, H-k] which is the max grade one can receive 
    with the remaining time and projects.
    
        \begin{equation}
            A(i,h)=\begin{cases}
         0, & \text{if $j=0$}.\\
        max_{0 \leq k \leq H}(A[i-1,H-k] + f_i(k)), & \text{otherwise}.
        \end{cases}
        \end{equation}
  
    To maximize the average grade over n courses we will do max $\frac{(f_1(h1) + f_2(h2) ... + f_n(hn))}{n}$ which is equivalent to maximizing $\sum_{i = 1}^ {n} f_i(hi)$
  
    \begin{verbatim}
        BottomUp(n, H):
            for h = 0 to H:
                A(0, h) = 0
            for i = 0 to n:
                A(i, 0) = f_i(0)
            
            for i = 1 to n:
                max = 0
                for j = 0 to H:
                    for k = 0 to j
                        g = A(i-1, H-k) + f_i(k)
                        if (g > max):
                            max = g
                    A(i, j) = max
            return A(n, H) /n
    \end{verbatim}
    
    \item[b.] \begin{verbatim}
        Augmented(n, H):
            for h = 0 to H:
                A(0, h) = 0
            for i = 0 to n:
                A(i, 0) = f_i(0)
            
            for i = 1 to n:
                max = 0
                for j = 0 to H:
                    for k = 0 to j
                        g = A(i-1, H-k) + f_i(k)
                        if (g > max):
                            max = g
                            hours[i] = k
                    A(i, j) = max
            return hours and A(n, H) /n
                    
        
    \end{verbatim}
\end{enumerate}
    
\section*{Question 2}
\begin{enumerate}
    \item[a.] The subproblem of this question is defined as:
    cost(i,j): The cost of connecting train carts i to j. The Bellman equation is as follows:

    \begin{equation}
        Cost(i,j)=\begin{cases}
    0, & \text{if $i=j$}.\\
    min_{i \leq k < j}(cost(i,k) + cost(k+1, j) + \\min((w_{i} + ... + w_{k})^{1/2},(w_{k+1} + ... + w_{j})^{1/2})), & \text{if $i < j$}.
    \end{cases}
    \end{equation}

    Pseudocode:
    \begin{verbatim}
        BottomUp(n, W):
            for i = 1 to n:
                cost(i, i) = 0
            
            for l = 1 to n-1:
                for i = 1 to n-1:
                    j = i + l
                    if j <= n:
                        min_found = inf
                        for k = i to j-1:
                            found = min(cost(i,k) + cost(k+1,j) + 
                            min(w[i:k]^1/2, w[k+1:j]^1/2))

                            if found < min_found:
                                min_found = found

                    cost(i,j) = min_found
            return cost(1,n)
    \end{verbatim}

    Time Complexity: O($n^3$) since we have 3 nested for loops 

    \item[b. ] To augment the algorithm so that it also outputs an optimal order
    of train car connections, we can add to the memory part of the algorithmn by
    adding another array that stores the order. The order is an array that stores
    pairs in the order in which the train carts are connected. 
    Ex: [[1,2],[3,4],[5,6],[2,3],[4,5]] 
    
    Pseudocode:
    \begin{verbatim}
        BottomUp(n, W):
            for i = 1 to n:
                cost(i, i) = 0
                order(i,i) = []
            
            for l = 1 to n-1:
                for i = 1 to n-1:
                    j = i + l
                    if j <= n:
                        min_found = inf
                        for k = i to j-1:
                            found = min(cost(i,k) + cost(k+1,j) + 
                            min(w[i:k]^1/2, w[k+1:j]^1/2))

                            found_order = order(i,k) + order(k+1,j) + [[k,k+1]]

                            if found < min_found:
                                min_found = found
                                min_order = found_order

                    cost(i,j) = min_found
                    order(i,j) = min_order
            return cost(1,n)
    \end{verbatim}

 \end{enumerate}

\section*{Question 3} 
\begin{enumerate}
    \item[a.] Let y = Green\\
              Let x = El\\
              Let z = Greelen\\
 
              The greedy algorithmn would take the `Gree' portion of z and remove that from y such that y = en.
              What remains is x = el, y = en, z = len, and from there, the greedy algorithm cannot continue further
              and must return false.

    \item[b.] The subproblem of this question is defined as IsInterleaving(i, j): True iff $z[1] ... z[i+j]$ is an interleaving of $x[1] ... x[i]$ and $y[1] ... y[j]$. The Bellman equation is as follows:
    
    \begin{equation*}
        \text{interleaving}(i,j)=\begin{cases}
    \text{True}, & \text{if $i = 0$ and $j = 1$} \\ & \text{and $z[1] = y[1]$} \\ & \text{OR} \\ & \text{if $i = 1$ and $j = 0$} \\ & \text{and $z[1] = x[1]$}.\\ \\
    z[i+j] == x[i] \newline \text{ and interleaving}(i-1, j) \\ \text{OR} & \text{otherwise}. \\ z[i+j] == y[j] \text{ and interleaving}(i, j-1)
    \end{cases}
    \end{equation*}
    
    Let the length of $x = n$ and the length of $y = m$. Psuedocode: 
    \begin{verbatim}
    IsInterleaving(x, y, z)
        n = |x|
        m = |y|
        In is 2dArray [i, j] of Bools.
            for i = 1 to n
            and for j = 1 to m
        
        Initialize In to False at every index
        
        In[0, 1] = z[1] == y[1]
        In[1, 0] = z[1] == x[1]
        
        for i = 1 to n
            for j = 1 to m
                In[i, j] = (z[i+j] == x[i] and In[i-1, j]) or \
                    (z[i+j] == y[j] and In[i, j-1])
        return In[n, m]
            
    \end{verbatim}
    The running time of this program is $\mathcal{O}(nm)$, as initializing the array takes $nm$ time, and the second loop also takes $nm$ time.

\end{enumerate}

\end{document}