\documentclass[12pt]{article}
\usepackage{dsfont}
\usepackage{amsmath}
\usepackage[margin=2.5cm]{geometry}
\usepackage{amssymb}
\usepackage{mathtools}

\DeclarePairedDelimiter\ceil{\lceil}{\rceil}
\DeclarePairedDelimiter\floor{\lfloor}{\rfloor}

\usepackage{algorithm}
\usepackage[noend]{algpseudocode}
\usepackage{pgf}
\usepackage{tikz}
\usepackage{tikz-qtree}

\title{\textbf{CSC373 - Problem Set 3}}
\author{Authors: Luke Bacchus, Naslin Rahman, Zhuoqian Li}
\date{\today}

\begin{document}
\maketitle
\section*{Question 1}
\begin{enumerate}
    \item[a.] Let A(i, h) = maximum grade one can receive on i projects if they spend h number of hours on it. The Bellman equation is as follows. TODO:correctness
    
        \begin{equation}
            A(i,h)=\begin{cases}
         0, & \text{if $j=0$}.\\
        max_{0 \leq k \leq H}(A[i-1,H-k] + f_i(k)), & \text{otherwise}.
        \end{cases}
        \end{equation}
  
    To maximize the average grade over n courses we will do max $\frac{(f_1(h1) + f_2(h2) ... + f_n(hn))}{n}$ which is equivalent to maximizing $\sum_{i = 1}^ {n} f_i(hi)$
  
    \begin{verbatim}
        BottomUp(n, H):
            for h = 0 to H:
                A(0, h) = 0
            
            for i = 1 to n:
                max = 0
                for j = 0 to H:
                    for k = 0 to j
                        g = A(i-1, H-k) + f_i(k)
                        if (g > max):
                            max = g
                    A(i, j) = max
            return A(n, H) /n
    \end{verbatim}
    
    \item[b.] \begin{verbatim}
        Augmented(n, H):
            for h = 0 to H:
                A(0, h) = 0
            
            for i = 1 to n:
                max = 0
                for j = 0 to H:
                    for k = 0 to j
                        g = A(i-1, H-k) + f_i(k)
                        if (g > max):
                            max = g
                            hours(i) = k
                    A(i, j) = max
            return hours
                    
        
    \end{verbatim}
    
    
\end{enumerate}
    
\section*{Question 2}

\section*{Question 3} 
\begin{enumerate}
    \item[a.] Let y = Green\\
              Let x = El\\
              Let z = Greelen\\
 
              The greedy algorithmn would take the 'Gree' portion of z and remove that from y such that y = en.
              What remains is x = el, y = en, z = len, and from there, the greedy algorithm cannot continue further
              and must return false.


\end{enumerate}

\end{document}