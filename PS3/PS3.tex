\documentclass[12pt]{article}
\usepackage{dsfont}
\usepackage{amsmath}
\usepackage[margin=2.5cm]{geometry}
\usepackage{amssymb}
\usepackage{mathtools}

\DeclarePairedDelimiter\ceil{\lceil}{\rceil}
\DeclarePairedDelimiter\floor{\lfloor}{\rfloor}

\usepackage{algorithm}
\usepackage[noend]{algpseudocode}
\usepackage{pgf}
\usepackage{tikz}
\usepackage{tikz-qtree}

\title{\textbf{CSC373 - Problem Set 3}}
\author{Authors: Luke Bacchus, Naslin Rahman, Zhuoqian Li}
\date{\today}

\begin{document}
\maketitle
\section*{Question 1}
\begin{enumerate}
    \item[a.] \begin{equation}
            A(j,H)=\begin{cases}
         0, & \text{if $j=0$}.\\
        max_{0 \leq k \leq H}(A[j-1,H-k] + f_j(k)), & \text{otherwise}.
        \end{cases}
        \end{equation}
  
    To maximize the average grade over n courses we will do max $\frac{(f_1(h1) + f_2(h2) ... + f_n(hn))}{n}$ which is equivalent to $\sum_{i = 1}^ {n} f_i(hi)$
  
    \begin{verbatim}
        BottomUp(n, H):
             for i 1 to n
    \end{verbatim}
\end{enumerate}
    
\section*{Question 2}

\section*{Question 3} 
\begin{enumerate}
    \item[a.] Let y = Green\\
              Let x = El\\
              Let z = Greelen\\
 
              The greedy algorithmn would take the 'Gree' portion of z and remove that from y such that y = en.
              What remains is x = el, y = en, z = len, and from there, the greedy algorithm cannot continue further
              and must return false.


\end{enumerate}

\end{document}