\documentclass[12pt]{article}
\usepackage{dsfont}
\usepackage{amsmath}
\usepackage[margin=2.5cm]{geometry}
\usepackage{amssymb}
\usepackage{mathtools}

\DeclarePairedDelimiter\ceil{\lceil}{\rceil}
\DeclarePairedDelimiter\floor{\lfloor}{\rfloor}

\usepackage{algorithm}
\usepackage[noend]{algpseudocode}
\usepackage{pgf}
\usepackage{tikz}
\usepackage{tikz-qtree}

\title{\textbf{CSC373 - Problem Set 2}}
\author{Authors: Luke Bacchus, Naslin Rahman, Zhuoqian Li}
\date{\today}

\begin{document}
\maketitle
\section*{Question 1}
This algorithm is very similar to the Interval scheduling problem done in class except we have an circle of problems (similar to a clock). We can apply the same algorithm if there is only one job that intersects wuth the 12 am time slot. So we do this for each each one of these jobs.
 
 The algorithm goes as follows:
 \begin{enumerate}
     \item For all intervals that contain the 12am slot put them into a set called S
     \item Sort the remaining intervals in order of earliest finish time (EFT)
     \item Create a new set A and select a job s from S and insert into the sorted intervals in step 2
     \item for each interval i in sorted order, if it is compatible with A then add it to A. 
     \item Repeat steps 3 and 4 for each job in S 
     \item Now for each set we created, return the one with greatest length
 \end{enumerate}

The run time of this algorithm is O(n+ cnlogn + $cn^2$ + n) which is $O(n^2)$

This question was written and read by  Luke, Naslin, and Zhuoqian.
\newpage

\section*{Question 2}
\begin{enumerate}
    \item[a.] Our goal is to find the easiest trip which is the trip with the minimum toughness. The toughness 
    of a trip is the greatest degree of difficulty among all portages that trip. Originally, in Dijkstra’s 
    algorithmn, we have d(v) = the sum of the weights between nodes/lakes on the path to node v. However, 
    the proposed modification is to assign d(v) as the max of the weights (aka "difficulties") of the portages 
    between lakes. \\

    Below is the pseudocode:

    \begin{verbatim}
        function find(s, t, L, n):
            R = [s]
            d[s] = 0
            V = [1,2,...,n]

            for v = 1 to n:
                if v != s:
                    if v in L[s]:
                        v_weight = L[s][v][1]
                        R.append(v)
                        d[v] = v_weight
                        p[v] = s
                    else:
                        R.append(inf)
                        d[v] = inf
                        p[v] = nil
            
            while R != V:
                not_R = V - R
                not_d = []

                for v in not_R:
                    not_d[v] = d[v]
        
                u = not_d.get_index(min(not_d))
                R.append(u)

                for v = 1 to n:
                    if v != u and v in not_R:
                        if v in L[u]:
                            if max(d[u], L[u][v][1]) < d[v]: 
                                d(v) =  max(d[u], L[u][v][1])
                                p(v) = u

            return d(t)
    \end{verbatim}

    The first for loop would have a runtime of c*n, where c is a constant. The while loop would take at most
    n iterations, while the first for loop within the while loop would take m runtime, where $m <= n$. The second
    for loop in the while loop would take at most n iterations. Thus the worst case runtime of the modified
    algorithm is $O(c + cn + m + n^2)$ and thus $O(n^2)$.

    \item[b.] In Dijkstra’s original algorithm, d(v) is a sum of the weights. It assumes that when a new 
    a new node, u, is added to R, for all nodes, v, in  V - R whose R-path contain u, 
    $d(v) = min(d(u) + w_uv, d(v))$. This relies on non negative weights since it relies on the shortest path
    to v being either the original path $s-> v$ or the path  $s->u$ + v. Negative weights would allow this 
    assumption to be false since a negative weight connected to v elsewhere could lessen the weight of another
    path, not mentioned previously, making it possible to be the shortest path to v.\\

    This assumption is not necessary since our modification takes the max weight of all the portages on the path.
    Thus, negative weights would not impact the outcome.\\
\end{enumerate}

This question was written and read by  Luke, Naslin, and Zhuoqian.
\newpage
\section*{Question 3} 
\begin{enumerate}
    \item[a.] Consider the actor heights of $900, 30, 15$, and the costume lengths of $50, 10, 5$. The optimal solution is matching $900 : 50$, $30 : 10$, and $15 : 5$ for a cost of $850 + 20 + 10 = 880$. However, Bombazzino's algorithm will match $15 : 10$, $30 : 50$, and $900 : 5$ for a cost of $5 + 20 + 895 = 920$. Thus, Bombazzino's algorithm is not optimal.
    \item[b.] Let $S_g$ be the solution that our greedy algorithm returns, and let $S_o$ be the optimal solution. We will show that the cost of $S_g = S_o$, therefore $S_g$ is optimal.
    \newline \,
    \newline
    To start, sort $S_g$ and $S_o$ by increasing actor height. Let $h_1$ to $h_k$ be the distinct heights $h_1 > h_2 > ... > h_k$ of the actors. Starting at $i = 1$, consider the actors of height $h_i$, and let $C_i$ be the set of costumes assigned to the actors of height $h_i$ in solution $S_g$. Let $D_i$ be the set of costumes assigned to the actors of height $h_i$ in $S_o$. If $C_i = D_i$ than the cost of the assignment to the actors of $h_i$ in $S_g$ is the same as the cost of the assignment to the actors of height $h_i$ in $S_o$ and this holds for $h_1$ up to $h_i$.
    \newline \,
    \newline
    Say that $C_i \neq D_i$. Let $a_i$ be an actor of height $h_i$ in $S_g$, such that in $S_o$, $a_i$ has a costume $c_j$ that is not a member of the set of costumes $C_i$. Therefore, the length of $c_j$ is less than the length of any costume in $C_i$, and there must also exist an actor $a_j$ with height less than $h_i$ such that $a_j$ has been given some costume $c_i$ where $c_i$ is in $C_i$ in $S_o$.
    \newline \,
    \newline
    Therefore, we have that the height of actor $a_i > a_j$, and the length of costume $c_i > c_j$. This is a contradiction, as then the optimal solution could be improved by swapping the costumes for $a_i$ and $a_j$ 
    (in an optimal solution, there cannot be a pair of actors $a_i$ $a_j$ where $a_i$ is taller than $a_j$ and $a_i$ is assigned a costume that is shorter than $a_j$'s costume. We know this since $ai > aj$ and $ci > cj$, which implies that $ai - cj > ai - ci$. This tells us that $cost(ai$ $cj) > cost(ai$ $ci)$, which must be suboptimal, as swapping the costumes for $a_i$ and $a_j$ will strictly improve the cost of the solution). Therefore, we see that in the optimal solution $C_i$ must be equal to $D_i$ for all $i$, and therefore the cost is identical.

\end{enumerate}

This question was written and read by  Luke, Naslin, and Zhuoqian.
\end{document}