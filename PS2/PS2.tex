\documentclass[12pt]{article}
\usepackage{dsfont}
\usepackage{amsmath}
\usepackage[margin=2.5cm]{geometry}
\usepackage{amssymb}
\usepackage{mathtools}

\DeclarePairedDelimiter\ceil{\lceil}{\rceil}
\DeclarePairedDelimiter\floor{\lfloor}{\rfloor}

\usepackage{algorithm}
\usepackage[noend]{algpseudocode}
\usepackage{pgf}
\usepackage{tikz}
\usepackage{tikz-qtree}

\title{\textbf{CSC373 - Problem Set 2}}
\author{Authors: Luke Bacchus, Naslin Rahman, Zhuoqian Li}
\date{\today}

\begin{document}
\maketitle
\section*{Question 1}

\section*{Question 2}
\begin{enumerate}
    \item[a.] Our goal is to find the easiest trip which is the trip with the minimum toughness. The toughness 
    of a trip is the greatest degree of difficulty among all portages that trip. Originally, in Dijkstra’s 
    algorithmn, we have d(v) = the sum of the weights between nodes/lakes on the path to node v. However, 
    the proposed modification is to assign d(v) as the max of the weights (aka "difficulties") of the portages 
    between lakes. \\

    Below is the pseudocode:

    \begin{verbatim}
        function find(s, t, L, n):
            R = [s]
            d[s] = 0
            V = [1,2,...,n]

            for v = 1 to n:
                if v != s:
                    if v in L[s]:
                        v_weight = L[s][v][1]
                        R.append(v)
                        d[v] = v_weight
                        p[v] = s
                    else:
                        R.append(inf)
                        d[v] = inf
                        p[v] = nil
            
            while R != V:
                not_R = V - R
                not_d = []

                for v in not_R:
                    not_d[v] = d[v]
        
                u = not_d.get_index(min(not_d))
                R.append(u)

                for v = 1 to n:
                    if v != u and v in not_R:
                        if v in L[u]:
                            if max(d[u], L[u][v][1]) < d[v]: \\TODO:Check if this logic follows?
                                d(v) =  max(d[u], L[u][v][1])
                                p(v) = u

            return d(t)
    \end{verbatim}
    TODO:   Give, and briefly justify, the complexity of your modified algorithm

    \item[b.] In Dijkstra’s original algorithm, d(v) is a sum of the weights. It assumes that when a new 
    a new node, u, is added to R, for all nodes, v, in  V - R whose R-path contain u, 
    $d(v) = min(d(u) + w_uv, d(v))$. This relies on non negative weights since it relies on the shortest path
    to v being either the original path s-> v or the path  s->u + v. Negative weights would allow this 
    assumption to be false since a negative weight connected to v elsewhere could lessen the weight of another
    path, not mentioned previously, making it possible to be the shortest path to v.\\

    This assumption is not necessary since our modification takes the max weight of all the portages on the path.
    Thus, negative weights would not impact the outcome.\\
\end{enumerate}


\end{document}